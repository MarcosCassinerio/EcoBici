% Options for packages loaded elsewhere
\PassOptionsToPackage{unicode}{hyperref}
\PassOptionsToPackage{hyphens}{url}
%
\documentclass[
]{article}
\usepackage{amsmath,amssymb}
\usepackage{lmodern}
\usepackage{ifxetex,ifluatex}
\ifnum 0\ifxetex 1\fi\ifluatex 1\fi=0 % if pdftex
  \usepackage[T1]{fontenc}
  \usepackage[utf8]{inputenc}
  \usepackage{textcomp} % provide euro and other symbols
\else % if luatex or xetex
  \usepackage{unicode-math}
  \defaultfontfeatures{Scale=MatchLowercase}
  \defaultfontfeatures[\rmfamily]{Ligatures=TeX,Scale=1}
\fi
% Use upquote if available, for straight quotes in verbatim environments
\IfFileExists{upquote.sty}{\usepackage{upquote}}{}
\IfFileExists{microtype.sty}{% use microtype if available
  \usepackage[]{microtype}
  \UseMicrotypeSet[protrusion]{basicmath} % disable protrusion for tt fonts
}{}
\makeatletter
\@ifundefined{KOMAClassName}{% if non-KOMA class
  \IfFileExists{parskip.sty}{%
    \usepackage{parskip}
  }{% else
    \setlength{\parindent}{0pt}
    \setlength{\parskip}{6pt plus 2pt minus 1pt}}
}{% if KOMA class
  \KOMAoptions{parskip=half}}
\makeatother
\usepackage{xcolor}
\IfFileExists{xurl.sty}{\usepackage{xurl}}{} % add URL line breaks if available
\IfFileExists{bookmark.sty}{\usepackage{bookmark}}{\usepackage{hyperref}}
\hypersetup{
  pdftitle={R Notebook},
  hidelinks,
  pdfcreator={LaTeX via pandoc}}
\urlstyle{same} % disable monospaced font for URLs
\usepackage[margin=1in]{geometry}
\usepackage{color}
\usepackage{fancyvrb}
\newcommand{\VerbBar}{|}
\newcommand{\VERB}{\Verb[commandchars=\\\{\}]}
\DefineVerbatimEnvironment{Highlighting}{Verbatim}{commandchars=\\\{\}}
% Add ',fontsize=\small' for more characters per line
\usepackage{framed}
\definecolor{shadecolor}{RGB}{248,248,248}
\newenvironment{Shaded}{\begin{snugshade}}{\end{snugshade}}
\newcommand{\AlertTok}[1]{\textcolor[rgb]{0.94,0.16,0.16}{#1}}
\newcommand{\AnnotationTok}[1]{\textcolor[rgb]{0.56,0.35,0.01}{\textbf{\textit{#1}}}}
\newcommand{\AttributeTok}[1]{\textcolor[rgb]{0.77,0.63,0.00}{#1}}
\newcommand{\BaseNTok}[1]{\textcolor[rgb]{0.00,0.00,0.81}{#1}}
\newcommand{\BuiltInTok}[1]{#1}
\newcommand{\CharTok}[1]{\textcolor[rgb]{0.31,0.60,0.02}{#1}}
\newcommand{\CommentTok}[1]{\textcolor[rgb]{0.56,0.35,0.01}{\textit{#1}}}
\newcommand{\CommentVarTok}[1]{\textcolor[rgb]{0.56,0.35,0.01}{\textbf{\textit{#1}}}}
\newcommand{\ConstantTok}[1]{\textcolor[rgb]{0.00,0.00,0.00}{#1}}
\newcommand{\ControlFlowTok}[1]{\textcolor[rgb]{0.13,0.29,0.53}{\textbf{#1}}}
\newcommand{\DataTypeTok}[1]{\textcolor[rgb]{0.13,0.29,0.53}{#1}}
\newcommand{\DecValTok}[1]{\textcolor[rgb]{0.00,0.00,0.81}{#1}}
\newcommand{\DocumentationTok}[1]{\textcolor[rgb]{0.56,0.35,0.01}{\textbf{\textit{#1}}}}
\newcommand{\ErrorTok}[1]{\textcolor[rgb]{0.64,0.00,0.00}{\textbf{#1}}}
\newcommand{\ExtensionTok}[1]{#1}
\newcommand{\FloatTok}[1]{\textcolor[rgb]{0.00,0.00,0.81}{#1}}
\newcommand{\FunctionTok}[1]{\textcolor[rgb]{0.00,0.00,0.00}{#1}}
\newcommand{\ImportTok}[1]{#1}
\newcommand{\InformationTok}[1]{\textcolor[rgb]{0.56,0.35,0.01}{\textbf{\textit{#1}}}}
\newcommand{\KeywordTok}[1]{\textcolor[rgb]{0.13,0.29,0.53}{\textbf{#1}}}
\newcommand{\NormalTok}[1]{#1}
\newcommand{\OperatorTok}[1]{\textcolor[rgb]{0.81,0.36,0.00}{\textbf{#1}}}
\newcommand{\OtherTok}[1]{\textcolor[rgb]{0.56,0.35,0.01}{#1}}
\newcommand{\PreprocessorTok}[1]{\textcolor[rgb]{0.56,0.35,0.01}{\textit{#1}}}
\newcommand{\RegionMarkerTok}[1]{#1}
\newcommand{\SpecialCharTok}[1]{\textcolor[rgb]{0.00,0.00,0.00}{#1}}
\newcommand{\SpecialStringTok}[1]{\textcolor[rgb]{0.31,0.60,0.02}{#1}}
\newcommand{\StringTok}[1]{\textcolor[rgb]{0.31,0.60,0.02}{#1}}
\newcommand{\VariableTok}[1]{\textcolor[rgb]{0.00,0.00,0.00}{#1}}
\newcommand{\VerbatimStringTok}[1]{\textcolor[rgb]{0.31,0.60,0.02}{#1}}
\newcommand{\WarningTok}[1]{\textcolor[rgb]{0.56,0.35,0.01}{\textbf{\textit{#1}}}}
\usepackage{graphicx}
\makeatletter
\def\maxwidth{\ifdim\Gin@nat@width>\linewidth\linewidth\else\Gin@nat@width\fi}
\def\maxheight{\ifdim\Gin@nat@height>\textheight\textheight\else\Gin@nat@height\fi}
\makeatother
% Scale images if necessary, so that they will not overflow the page
% margins by default, and it is still possible to overwrite the defaults
% using explicit options in \includegraphics[width, height, ...]{}
\setkeys{Gin}{width=\maxwidth,height=\maxheight,keepaspectratio}
% Set default figure placement to htbp
\makeatletter
\def\fps@figure{htbp}
\makeatother
\setlength{\emergencystretch}{3em} % prevent overfull lines
\providecommand{\tightlist}{%
  \setlength{\itemsep}{0pt}\setlength{\parskip}{0pt}}
\setcounter{secnumdepth}{-\maxdimen} % remove section numbering
\ifluatex
  \usepackage{selnolig}  % disable illegal ligatures
\fi

\title{R Notebook}
\author{}
\date{\vspace{-2.5em}}

\begin{document}
\maketitle

\begin{Shaded}
\begin{Highlighting}[]
\NormalTok{usuarios }\OtherTok{=} \FunctionTok{read.table}\NormalTok{(}\StringTok{"usuarios5.csv"}\NormalTok{, }\AttributeTok{header =} \ConstantTok{TRUE}\NormalTok{, }\AttributeTok{sep =} \StringTok{","}\NormalTok{)}
\NormalTok{recorridos }\OtherTok{=} \FunctionTok{read.table}\NormalTok{(}\StringTok{"recorridos5.csv"}\NormalTok{, }\AttributeTok{header =} \ConstantTok{TRUE}\NormalTok{, }\AttributeTok{sep =} \StringTok{","}\NormalTok{)}
\NormalTok{usuarios}\SpecialCharTok{$}\NormalTok{genero\_usuario }\OtherTok{=} \FunctionTok{as.factor}\NormalTok{(usuarios}\SpecialCharTok{$}\NormalTok{genero\_usuario)}
\NormalTok{usuarios\_recorridos }\OtherTok{=} \FunctionTok{merge}\NormalTok{(usuarios, recorridos)}

\FunctionTok{attach}\NormalTok{(recorridos)}

\NormalTok{dias }\OtherTok{=} \FunctionTok{factor}\NormalTok{(recorridos}\SpecialCharTok{$}\NormalTok{dia, }\AttributeTok{levels =} \FunctionTok{c}\NormalTok{(}\StringTok{"Domingo"}\NormalTok{, }\StringTok{"Lunes"}\NormalTok{, }\StringTok{"Martes"}\NormalTok{, }\StringTok{"Miércoles"}\NormalTok{, }\StringTok{"Jueves"}\NormalTok{, }\StringTok{"Viernes"}\NormalTok{, }\StringTok{"Sábado"}\NormalTok{))}
\NormalTok{usuarios\_recorridos}\SpecialCharTok{$}\NormalTok{dia }\OtherTok{=} \FunctionTok{factor}\NormalTok{(dias)}

\NormalTok{direcciones }\OtherTok{=} \FunctionTok{as.factor}\NormalTok{(}\FunctionTok{c}\NormalTok{(direccion\_estacion\_origen, direccion\_estacion\_destino))}

\NormalTok{vn }\OtherTok{=} \FunctionTok{sort}\NormalTok{(}\FunctionTok{table}\NormalTok{(direcciones), }\AttributeTok{decreasing =} \ConstantTok{TRUE}\NormalTok{)}
\NormalTok{vn2 }\OtherTok{=} \FunctionTok{sort}\NormalTok{(vn, }\AttributeTok{decreasing =} \ConstantTok{TRUE}\NormalTok{)[}\DecValTok{1}\SpecialCharTok{:}\DecValTok{10}\NormalTok{]}
\end{Highlighting}
\end{Shaded}

Top 10 estaciones

\begin{verbatim}
par(mar=c(17,4,4,2))
barplot(vn2, las=2, ylim = c(0, 40), ylab = "Usos", col = rainbow(10))
\end{verbatim}

Uso de Estaciones

\begin{verbatim}
par(mar=c(5,4,4,2))
plot(vn, ylim = c(0, 40), xaxt='n', xlab = "Paradas", ylab = "Usos", col = rainbow(1200))
\end{verbatim}

Viajes cada Dia

\begin{verbatim}
viajes_semana = table(usuarios_recorridos$dia)
barplot(viajes_semana, xlab = "Dia", ylab = "Viajes", names = levels(dias), col = rainbow(7), beside = TRUE)
\end{verbatim}

Viajes por Sexo cada Dia

\begin{verbatim}
viajes_sexo_semana = table(usuarios_recorridos$genero_usuario, usuarios_recorridos$dia)
barplot(viajes_sexo_semana, xlab = "Dia", ylab = "Viajes", names = levels(dias), col = c("#FF8989", "#A6F6F1", "#CEFA8A"), beside = TRUE)
legend("topright", legend = c("Mujeres", "Hombres", "Otros"), fill = c("#FF8989", "#A6F6F1", "#CEFA8A"))
\end{verbatim}

Usuarios por Sexo

\begin{verbatim}
usuarios_generos = table(usuarios$genero_usuario)
porcentajes_generos = usuarios_generos
porcentajes_generos = paste(c("Mujeres", "Hombres", "Otros"), porcentajes_generos)
porcentajes_generos = paste(porcentajes_generos, "%", sep = "")
pie(usuarios_generos, labels = porcentajes_generos, col = c("#FF8989", "#A6F6F1", "#CEFA8A"))
\end{verbatim}

Viajes Sexos

\begin{verbatim}
viajes_sexo = table(usuarios_recorridos$genero_usuario)
porcentajes_viajes_sexo = round((viajes_sexo * 100)/417)
porcentajes_viajes_sexo = paste(c("Mujeres", "Hombres", "Otros"), porcentajes_viajes_sexo)
porcentajes_viajes_sexo = paste(porcentajes_viajes_sexo, "%", sep = "")
pie(viajes_sexo, labels = porcentajes_viajes_sexo, col = c("#FF8989", "#A6F6F1", "#CEFA8A"))
\end{verbatim}

Usuarios por Edad (tallo y hoja, e histograma)

\begin{verbatim}
stem(usuarios$edad_usuario, atom = 10)
hist(usuarios$edad_usuario, main = "", xlab = "Edades", ylab = "Cantidad", col = "purple", ylim = c(0, 25), xaxp = c(10, 65, 11))
\end{verbatim}

Viajes Edades

\begin{verbatim}
hist(usuarios_recorridos$edad_usuario, main = "", xlab = "Edades", ylab = "Cantidad", col = "purple", ylim = c(0, 140), xaxp = c(10, 65, 11))
\end{verbatim}

Tabla viajes Edades

\begin{verbatim}
intervalos_edades = cut(usuarios_recorridos$edad_usuario, breaks = c(15, 20, 25, 30, 35, 40, 45, 50, 55, 60, 65))
frecuencia_absoluta = table(intervalos_edades)
frecuencia_relativa = prop.table(frecuencia_absoluta)
porcentaje = frecuencia_relativa * 100
frecuencia_absoluta_acumulada = cumsum(frecuencia_absoluta)
frecuencia_relativa_acumulada = cumsum(frecuencia_relativa)
porcentaje_acumulado = cumsum(porcentaje)
tabla_edad = cbind(frecuencia_absoluta, frecuencia_relativa, porcentaje, frecuencia_absoluta_acumulada, frecuencia_relativa_acumulada, porcentaje_acumulado)
\end{verbatim}

Analisis: Paradas mas usadas Uso de rodados por edad

\end{document}
